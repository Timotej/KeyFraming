\documentclass[paper=a4, fontsize=11pt]{scrartcl} % A4 paper and 11pt font size

\usepackage[utf8]{inputenc}
\usepackage[T1]{fontenc} % Use 8-bit encoding that has 256 glyphs
\usepackage{fourier} % Use the Adobe Utopia font for the document - comment this line to return to the LaTeX default
\usepackage[english]{babel} % English language/hyphenation
\usepackage{amsmath,amsfonts,amsthm} % Math packages

\usepackage{lipsum} % Used for inserting dummy 'Lorem ipsum' text into the template

\usepackage{sectsty} % Allows customizing section commands
\allsectionsfont{\centering \normalfont\scshape} % Make all sections centered, the default font and small caps

\usepackage{fancyhdr} % Custom headers and footers
\pagestyle{fancyplain} % Makes all pages in the document conform to the custom headers and footers
\fancyhead{} % No page header - if you want one, create it in the same way as the footers below
\fancyfoot[L]{} % Empty left footer
\fancyfoot[C]{} % Empty center footer
\fancyfoot[R]{\thepage} % Page numbering for right footer
\renewcommand{\headrulewidth}{0pt} % Remove header underlines
\renewcommand{\footrulewidth}{0pt} % Remove footer underlines
\setlength{\headheight}{13.6pt} % Customize the height of the header

\numberwithin{equation}{section} % Number equations within sections (i.e. 1.1, 1.2, 2.1, 2.2 instead of 1, 2, 3, 4)
\numberwithin{figure}{section} % Number figures within sections (i.e. 1.1, 1.2, 2.1, 2.2 instead of 1, 2, 3, 4)
\numberwithin{table}{section} % Number tables within sections (i.e. 1.1, 1.2, 2.1, 2.2 instead of 1, 2, 3, 4)

\setlength\parindent{0pt} % Removes all indentation from paragraphs - comment this line for an assignment with lots of text

%----------------------------------------------------------------------------------------
%	TITLE SECTION
%----------------------------------------------------------------------------------------

\newcommand{\horrule}[1]{\rule{\linewidth}{#1}} % Create horizontal rule command with 1 argument of height

\title{	
%\normalfont \normalsize 
%\textsc{university, school or department name} \\ [25pt] % Your university, school and/or department name(s)
%\horrule{0.5pt} \\[0.4cm] % Thin top horizontal rule
%\huge 
Dynamika tuhých telies, definícia problému, rovnice pohybu (4 ODE), rýchlosť, zrýchlenie, uhľová rýchlosť a uhľové zrýchlenie, matica hybnosti (matica inercie) \\ % The assignment title
%\horrule{2pt} \\[0.5cm] % Thick bottom horizontal rule
}

\author{Timotej Krajči, Peter Šulík} % Your name

\date{\normalsize\today} % Today's date or a custom date

\begin{document}

\maketitle % Print the title

%----------------------------------------------------------------------------------------
%	PROBLEM 1
%----------------------------------------------------------------------------------------

\section{Tuhé teleso}

Tuhé teleso je také, ktorého tvar sa nikdy počas simulácie nezdeformuje. Kvôli tuhosti je
celkový pohyb telesa zložený z lineárneho pohybu ťažiska telesa a s rotácie telesa okolo
svojho ťažiska.

\section{Rovnice pohybu}
$$ TODO ROVNICE $$

\section{Rýchlosť}
Rýchlosť je deriváciou pozície podľa času: $$v(t) = c'(t)$$

\section{Zrýchlenie}
Zrýchlenie je definované ako derivácia rýchlosti podľa času:
$$ a(t) = v'(t) = (M^-1P)' = M^-1f$$

\section{Uhlová rýchlosť}
Uhľová rýchlosť je vektor, ktorý je rovnobežný s osou rotácie, ktorého dĺžka je  rovná rýchlosti rotácie. Rýchlosť rotácie je počet radiánov otočenia okolo osi za sekundu.
$$ q'(t) = \frac{1}{2}Q(t)\omega(t)$$
$$ TODO ROVNICE $$

\section{Uhlové zrýchlenie}
Uhľové zrýchlenie je definované ako derivácia uhľovej rýchlosťi v čase.
$$\alpha(t) = \omega'(t) = (J^{-1}L)' = J^{-1}L + J^{-1}L' = J^{-1}\omega \times J\omega + J^{-1} \tau $$

kde $\tau$ je krútiaci moment.

\section{Matica hybnosti}
Ak tuhé teleso pokladáme za množinu častíc s polohami $p_i$ a hmotnosťami $m_i$ tak tažisko
telesa je definované ako:

$$ c = \frac{\sum m_i p_i}{M} $$
$$ M = \sum m_i $$

Relatívna poloha i-tej častice je potom $p_i = c + r_i$ a absolútna poloha je $p_i = c + Rr_{0i}$ ,
kde R zodpovedá matici rotácie a $r_{0i}$ je poloha i-tej častice na začiatku. \\
Inerčný tenzor potom vieme definovať ako:
$$TODO VZOREC$$
Označenie r × značí antisymetrickú maticu vektorového súčinu:
$$TODO VZOREC$$
Narozdieľ od hmotnosti je inerčný senzor závislý od času. Keďže teleso sa nikdy nede-
formuje, môžeme výpočet inerčného tenzora iba na začiatku a jeho zmenu dopočítavat ’
pomocou matice rotácie:
$$J_0 = - \sum m_i r_{0i}^{\times} r_{0i}^{\times} $$
$$ J = RJ_{0}R^{T} $$
$$ J^{-1} = RJ_{0}^{-1}R^{T}$$



\begin{align} 
\begin{split}
(x+y)^3 	&= (x+y)^2(x+y)\\
&=(x^2+2xy+y^2)(x+y)\\
&=(x^3+2x^2y+xy^2) + (x^2y+2xy^2+y^3)\\
&=x^3+3x^2y+3xy^2+y^3
\end{split}					
\end{align}


%------------------------------------------------



%----------------------------------------------------------------------------------------
%	PROBLEM 1
%----------------------------------------------------------------------------------------

\section{Problem title}

\lipsum[2] % Dummy text

\begin{align} 
\begin{split}
(x+y)^3 	&= (x+y)^2(x+y)\\
&=(x^2+2xy+y^2)(x+y)\\
&=(x^3+2x^2y+xy^2) + (x^2y+2xy^2+y^3)\\
&=x^3+3x^2y+3xy^2+y^3
\end{split}					
\end{align}

Tuhé teleso je také, ktorého tvar sa nikdy počas simulácie nezdeformuje. Kvôli tuhosti je
celkový pohyb telesa zložený z lineárneho pohybu ťažiska telesa a s rotácie telesa okolo
svojho ťažiska.

%------------------------------------------------





































\subsection{Heading on level 2 (subsection)}

Lorem ipsum dolor sit amet, consectetuer adipiscing elit. 
\begin{align}
A = 
\begin{bmatrix}
A_{11} & A_{21} \\
A_{21} & A_{22}
\end{bmatrix}
\end{align}
Aenean commodo ligula eget dolor. Aenean massa. Cum sociis natoque penatibus et magnis dis parturient montes, nascetur ridiculus mus. Donec quam felis, ultricies nec, pellentesque eu, pretium quis, sem.

%------------------------------------------------

\subsubsection{Heading on level 3 (subsubsection)}

\lipsum[3] % Dummy text

\paragraph{Heading on level 4 (paragraph)}

\lipsum[6] % Dummy text

%----------------------------------------------------------------------------------------
%	PROBLEM 2
%----------------------------------------------------------------------------------------

\section{Lists}

%------------------------------------------------

\subsection{Example of list (3*itemize)}
\begin{itemize}
	\item First item in a list 
		\begin{itemize}
		\item First item in a list 
			\begin{itemize}
			\item First item in a list 
			\item Second item in a list 
			\end{itemize}
		\item Second item in a list 
		\end{itemize}
	\item Second item in a list 
\end{itemize}

%------------------------------------------------

\subsection{Example of list (enumerate)}
\begin{enumerate}
\item First item in a list 
\item Second item in a list 
\item Third item in a list
\end{enumerate}

%----------------------------------------------------------------------------------------

\end{document}